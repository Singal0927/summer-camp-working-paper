\documentclass[lang=cn]{elegantpaper}


\title{\vspace{-2cm}中国银行体系金融风险跨机构传导及系统重要性识别\\{\large——基于BERT-wwm与高斯网络}}
\author{葛新杰}
\date{}
\usepackage{booktabs}
\usepackage{float}
\usepackage{geometry}
\usepackage{longtable}
\usepackage{bm}
\setmainfont{Times New Roman}
\setCJKmainfont{Songti SC}
\makeatletter
\@addtoreset{equation}{section}
\makeatother
\renewcommand{\theequation}{\arabic{section}.\arabic{equation}}
%\geometry{left=1in,right=0.75in,top=1in,bottom=1in}


\begin{document}
\maketitle
\makeatletter
  \def\@cite#1#2{\textsuperscript{[{#1\if@tempswa , #2\fi}]}}
\makeatother

\vspace{-1cm}
\begin{abstract}\vspace{-0.7cm}

本文基于BERT-wwm预训练模型提取互联网文本数据所反映的投资者情绪值,结合股票收益率和动态CoVaR作为系统性金融风险传染的三种渠道,运用高斯网络模型构建20家上市银行机构的关联网络,量化研究系统性金融风险传染渠道和效应,并构建指标体系以识别系统重要性银行机构。本文的实证结果和政策建议如下:(1)基于市场波动和尾部风险传染渠道的系统性金融风险传染效应最强,而通过投资者情绪传染渠道则相对较弱,监管部门更应关注银行机构尾部风险的变化,并重视投资者情绪对市场稳定的积极效应,抑制投资者非理性情绪所产生的追涨杀跌动能,以平熨市场的非理性波动;(2)我国上市银行机构关联网络具有“小世界现象”和“无标度特性”,即银行系统具备“稳健且脆弱”的特征,监管机构应进行差异化监管,从而高效降低金融风险的传染速度和范围;(3)中心性指标将部分股份制商业银行识别为系统性重要银行,由于股份制商业银行的业务过度表外化,从而形成了影子银行,因此在宏观审慎监管中应进一步限制股份制商业银行影子业务规模的快速扩张。
\keywords{系统性金融风险传染;银行机构关联网络;自然语言处理;概率图模型}
\end{abstract}

%摘要:本文基于BERT-wwm预训练模型提取互联网文本数据所反映的投资者情绪值,结合股票收益率和动态CoVaR作为系统性金融风险传染的三种渠道,运用高斯网络模型构建20家上市银行机构的关联网络,量化研究系统性金融风险传染渠道和效应,并构建指标体系以识别系统重要性银行机构。本文的实证结果和政策建议如下:(1)基于市场波动和尾部风险传染渠道的系统性金融风险传染效应最强,而通过投资者情绪传染渠道则相对较弱,监管部门更应关注银行机构尾部风险的变化,并重视投资者情绪对市场稳定的积极效应,抑制投资者非理性情绪所产生的追涨杀跌动能,以平熨市场的非理性波动;(2)我国上市银行机构关联网络具有“小世界现象”和“无标度特性”,即银行系统具备“稳健且脆弱”的特征,随机冲击影响低联通节点的概率较高,但违约概率较低,但若随机冲击影响了某个高联通节点,则银行系统整体将发生严重的系统性金融风险,因此监管机构应重点加强对中心银行机构的支持力度,进行差异化监管,从而高效降低金融风险的传染速度和范围;(3)由于国有大型商业银行是央行通过货币政策调控宏观经济并疏散风险的重要渠道,故其系统重要性较强,但部分股份制商业银行业务种类繁多,财务杠杆较高,同业业务的过度表外化形成了影子银行,具备风险生成银行和风险承受银行的双重特征,从而加速了金融风险跨行业、跨市场传递,同时中心性指标反映股份制商业银行拥有更强的风险传染能力,因此在宏观审慎监管中应进一步限制股份制商业银行影子业务规模的快速扩张。

\section{引言}
关联性是现代风险测度和管理的核心因素,是市场风险、信用风险、交易对手风险和系统性风险的重要特征,也是经济周期风险以及宏观经济风险的重要组成部分(Diebold et al.,2014)\cite{01}。高度关联性使负面冲击在金融机构之间不断传导扩散,增加了个别金融机构的损失风险向整个金融体系扩散的可能性,致使微小冲击具备较高的破坏水平和影响范围。2008年金融危机爆发,系统性金融风险受到金融机构、监管部分等多方主体的密切关注。此后,2010年的希腊主权债务危机、2015年我国股市崩盘、2017年P2P等民间借贷行业暴雷事件等均与系统性金融风险发生导致单个机构破产损失并引起金融体系流动性短缺有关。一方面,危机过后,相比于传统认知的“太大而不能倒”(too big to fail),社会各方提出了“太关联而不能倒”(too interconnected to fail)的概念,金融稳定委员会(Financial Stability Board,FSB)也将关联性纳入识别系统重要性银行的指标之一,金融体系过度关联受到国际社会和各国监管机构的广泛关注。另一方面,由于我国金融改革持续深化,金融创新不断涌现,金融机构之间以及金融机构内各部门的业务联系更加紧密,我国金融分业监管体制面临巨大压力。2021年底召开的中央经济工作会议以“稳增长”为主线,指出“要正确认识和把握防范化解重大风险”、“研究制定化解风险的政策”以及“完善金融风险处置机制”,党的十九大将防范化解重大风险排在三大攻坚战的首位,“十四五”规划也指出“维护金融安全,守住不发生系统性风险底线”。同时,在后疫情时代,金融波动持续存在,面对金融和经济形势持续下行的压力,各国央行普遍施行“超常规”的量化宽松政策,流动性泛滥,各国债务占国民生产总值的比重大幅升高,国际金融形势以及全球经济复苏尚不明朗,故防范化解系统性金融风险将成为各国央行等监管机构的重点任务。因此,将传统金融理论与前沿技术方法相结合,准确测度金融机构之间的关联水平,并深入研究我国银行体系的系统性金融风险传染渠道、效应以及特征,将有助于制定相应的监管政策,对我国“一行两会”分业监管体制的完善与创新具备重要意义。

%由于我国现阶段的融资方式仍以间接融资为主,银行是间接融资的重要参与主体,因此我国银行机构承担了较高的金融风险,当个别金融风险不断累积并交叉传染时,一个微小的冲击便可能引致系统性金融风险的发生。。

%本文余下内容包括:第二章综述系统性金融风险的内部成因、传染渠道以及分别基于统计计量和机器学习的研究方法;第三章介绍本文所采用的NLP模型、动态CoVaR的计算以及高斯网络模型;第四章为三种数据渠道的描述与处理;第五章是实证分析部分,包括渠道分析和系统重要性评估;第六章是结论与政策含义。

%2008年在华盛顿举办的G20峰会将次贷危机的成因归结于:经济高速增长时期,资本流动性日益提高并且此前十年保持长期稳定,市场参与者过度追求高收益,缺乏风险评估。同时,脆弱的保险业标准,不健全的风险管理行为,复杂且不透明的金融产品以及由此引发的过度影响,最终加剧了金融脆弱性。一些发达国家的决策者和监管机构没有充分意识到金融市场不断扩大的风险,未能施行有效监管或金融革新,从而引发了次贷危机。

对于系统性金融风险的成因,陶玲等(2016)\cite{02}将其内部因素总结为金融的脆弱性、市场的有限理性和资产价格的波动性等。一是金融市场具备内在脆弱性,Minsky(1985)\cite{03}提出的“金融不稳定假说”认为,经济理论不应该把金融因素当作外生冲击,并且不能将经济衰退归咎于中央银行的失职。金融不稳定性假说表明金融危机与金融本身是密切关联的,由于信用创造机构和借款人的特征所导致的金融不稳定性,致使金融本身也是产生金融危机的关键原因之一。Kregel(1997)\cite{04}从借款人和放款人的角度来解释金融脆弱性的成因。资金借贷双方的安全边界(Margins of Safety)是由预期现金收入决定的,金融脆弱性正是建立在安全边界发生变化的基础之上。当安全边界缩减到最低值时,即使现实与预期的最小偏差也将迫使公司偏离原定计划,以满足事先确定的净现金流入,最终造成债务-紧缩旋涡——价格下跌速度大于还债速度,致使实际债务上升。二是金融过度创新以及另类投资工具的不断发展,导致金融产品层层嵌套,加剧了金融风险,扩大了金融业的顺周期效应。三是系统重要性金融机构(SIFIs)成为系统性风险的主要来源,FSB et al.(2010)\cite{05}通过三个关键指标——规模、可替代性以及关联性对SIFIs进行定义,一旦SIFIs因经营不善或外部因素导致风险过度积累,则风险将通过SIFIs与其他金融机构的内在关联性传染。四是信息不对称所导致的道德风险。Corsetti et al.(1998)\cite{06}认为,1998年亚洲金融危机期间,即使市场的过度反映和羊群效应是汇率和资产价格暴跌的重要原因,但同时也反映出某些国家或地区的结构和政策扭曲。金融机构一方面缺乏必要的政府监管,另一方面得到了政府的隐性担保,从而金融业的风险不断聚集,最终爆发了亚洲金融危机。

系统性金融风险的传染渠道可大致分为真实联系渠道、信息渠道以及尾部风险渠道。一是通过真实联系渠道传染,包括经济贸易往来、信贷关联和产业链关联(杨子晖,2022)\cite{07}。二是通过信息渠道传染,在市场信息不对称的条件下,投资者情绪变化以及对资产风险偏好的转变等因素会导致“羊群行为”,并诱发风险传递与扩散现象。例如,王美今等(2004)\cite{08}通过改进噪声交易者模型(DSSW)发现投资者接收价格信号时所表现出的情绪是影响均衡价格的系统性因子。黄宏斌等(2016)\cite{09}的研究表明,处于各生命周期的企业应利用投资者情绪的变化来选择融资的时点和方式,以最大程度地缓解融资约束并降低流动性风险。张宗新等(2013)\cite{10}建立了“预期调整——投资者情绪——市场波动”的逻辑框架,实证分析表明,市场上信息的变化通过改变投资者的预期进而影响投资者情绪,并最终导致市场波动。三是尾部风险渠道传染,表现为投资收益可能偏离均值超过三个标准差。如Adrian et al.(2016)\cite{11}提出条件风险价值(CoVaR),通过计算金融机构分别处于正常和极端两种状态下,金融体系的风险差值来衡量系统性风险的尾部传染。杨子辉等(2018)\cite{12}通过预期损失指标(ES)来衡量金融市场的尾部风险,考察金融风险的跨部门传染效应,并运用网络关联指标,对金融系统整体及单个金融机构尾部风险的非线性特征进行分析。

%重写,参照金融研究那篇
系统性金融风险传染的现有测度方法以统计方法为主。一是20世纪90年代各国金融机构所采纳的在险价值(VaR)和预期损失指标(ES)。相较于传统风险度量指标,VaR适用于不同金融工具的风险测度问题(Jorion,1997)\cite{13},但无法正确衡量极端事件发生后的预期损失,在此基础上,一些文献提出ES指标,以测度条件期望损失值,并作为VaR的有效补充。然而,VaR与ES指标仅仅考察了个体风险,未能将单个机构与系统间的风险传染效应纳入模型。二是基于“资产组合式”的风险测度方法(杨子晖,2022)\cite{07},即通过考察个体对系统风险的贡献,来衡量系统性风险,如Acharya et al.(2017)\cite{14}所提出的边际期望损失(MSE)和SES指标等,但MSE指标忽略了金融网络的整体关联度,SES指标的准确性与基准事件的时间差呈负相关。三是基于尾部依赖性,考察巨灾事件下机构损失对系统损失的贡献度,以测度系统性金融风险的传染效应,典型方法为Adrian and Brunnermeier(2016)\cite{11}提出的条件风险价值(Conditional Value - at - Risk,CoVaR)。四是基于金融网络的拓扑结构来刻画金融机构间的关联网络和风险传染机制。例如,李政等(2016)\cite{15}研究发现我国金融机构的网络结构具备“小世界现象”和“无标度特性”等网络性质。方意等(2016)\cite{16}通过构建有向共同资产网络模型,发现风险承受银行往往与风险生成银行具备类似的资产结构,并且对风险枢纽银行所持有的易遭受外生冲击的资产数量加以限制,是抑制系统性风险在金融网络中传播的有效途径。杨子晖等(2020)\cite{17}通过对全球19个主要国家或地区的经济政策不确定性和系统性金融风险传导的关系展开研究,发现股票市场是风险的主要输出方,对外汇市场和经济政策不确定性具备较强的溢出效应,且境外金融市场会对我国金融市场产生显著的风险传染。宫晓莉等(2020)\cite{18}采用方差分解网络对我国上市金融机构的信息溢出进行研究,发现通货膨胀和财务杠杆会显著地强化系统性风险的外溢,银行间同业拆借利率与系统性风险外溢指标呈负相关。五是采用计算机仿真技术来模拟金融风险的在金融网络中的传染路径,马骏等(2021)\cite{19}使用我国上市银行的财报和市场数据刻画了银行在约束条件下的最优抛售行为,模拟了资产出售对金融风险在金融机构间传导路径的影响,研究发现,单个银行的最优行为可能会加剧风险传染,且风险的传染呈高度非线性。

%基于历史数据对金融机构的损失概率分布进行估计,计算在金融巨灾事件发生的条件下,系统性金融风险的传染效应,如CoVaR法和ES法等,但。二是采用未定权益法(CCA),CCA将Black-Scholes-Merton期权定价模型拓展应用于宏观金融风险分析,对五个宏观经济部门(居民、政府、非金融企业、金融机构和国外)的资产负债集合进行研究,并将资产负债表与金融信息相结合,构建出反映金融风险的资产负债信息,最后运用BS定价原理对风险债务进行定价。宫晓琳(2012)\cite{13}利用CCA建立了各部门的风险资产负债表,派生出财务危机距离、违约概率、预期损失净现值等风险指标,反映出2008年金融危机对我国各部门金融风险的影响:在杠杆率和波动率共同上升地作用下,金融风险急剧升高。也有学者对CCA模型进行了优化,以更好地刻画极端事件所具备的风险激增特征,如唐文进等(2017)\cite{14}将传统CCA模型的连续扩散假设放松为跳跃扩散假设,提出了跳跃未定权益分析模型,相比传统研究提前3至6个月预警系统性风险。三是通过计算夏普利值(Shapley)来度量系统性金融风险的传染效应,夏普利值属于合作对策问题,利用夏普利值可以实现对成员利益的合理分配,避免了平均主义。四是基于金融网络分析来刻画金融机构间的关联网络和风险传染。李政等(2016)\cite{15}研究发现我国金融机构的网络结构具备“小世界现象”和“无标度特性”等网络性质,且自2012年起,我国金融机构的关联性呈上升趋势。方意等(2016)\cite{16}以风险生成银行和风险承受银行为网络节点,以风险传染为边,构建了有向共同资产网络模型,研究发现,风险承受银行往往与风险生成银行具备类似的资产结构,并且对风险枢纽银行所持有的易遭受外生冲击的资产数量加以限制,是抑制系统性风险在金融网络中传播的有效途径。杨子晖等(2020)\cite{17}通过对全球19个主要国家或地区的经济政策不确定性和系统性金融风险传导的关系展开研究,发现股票市场是风险的主要输出方,对外汇市场和经济政策不确定性具备较强的溢出效应,且境外金融市场会对我国金融市场产生显著的风险传染,其中香港金融市场最易遭受冲击。宫晓莉等(2020)\cite{18}采用方差分解网络对我国上市金融机构的信息溢出进行研究,通货膨胀和财务杠杆会显著地强化系统性风险的外溢,银行间同业拆借利率与系统性风险外溢指标呈负相关。此外,还有学者采用计算机仿真技术来模拟金融风险的在金融网络中的传染路径,马骏等(2021)\cite{19}使用我国上市银行的财报和市场数据刻画了银行在约束条件下的最优抛售行为,模拟了资产出售对金融风险在金融机构间传导路径的影响,研究发现,单个银行的最优行为可能会加剧风险传染,且风险的传染呈高度非线性。

近年来,随着数据获取的难度降低、中央处理器与图形处理器的运行速度显著提高,一些学者将统计学习方法与机器学习技术结合起来,结合大数据技术共同研究金融风险在金融网络间的传染效应。党印等(2022)\cite{20}认为现阶段系统性金融风险的研究方法主要使用传统结构化金融数据,而诸如网络爬虫、文本挖掘等大数据方法为金融风险的研究提供了新的数据源——非结构化数据,此外网络分析法将金融体系视为不同金融机构基于关联关系所构成的网络,更加全面地考量金融风险在金融体系中的传染效应。Gandrud et al.(2015)\cite{21}运用核主成分分析法,对经济学智库(Economist Intelligence Unit,EIU)所发布的各国月度报告进行研究,将定性文本转换为连续的横截面时间序列,构建了实时、连续的银行系统压力测度指标,并将该指标应用于政治经济周期中的债务问题。Cerchiello et al.(2018)\cite{22}基于文档转换向量(Doc2Vec)的方法,将单词序列中所包含的文本数据映射到降维后的潜在语义空间,结合词向量与财务数据来训练神经网络,实证结果表明:文本数据的引入能够提高分类器的准确度,即更为精准地判断银行处于困境或者正常状态。Chen et al.(2020)\cite{23}将文本数据与多种机器学习方法相结合,用于不同金融危机的识别和预测,研究发现:文本数据有助于降低测试集的假阴率和假阳率,当金融危机愈发严重时,该效果更为显著。Nyman et al.(2021)\cite{24}使用文本分析算法,从定性描述中提取出定量数据,并证明了在低频状态下文本情绪的转变与金融市场事件密切相关,在高频下,文本数据一方面能够引导市场情绪、预期和波动,另一方面可以为经济发展提供有益信息。

总体来看,目前关于系统性金融风险的测度方法大都是以统计方法为主,神经网络、自然语言处理等大数据方法的应用并不多见。同时,金融机构的财务数据、股票收益率等结构化数据仍是系统性金融风险研究所关注的重点,但缺乏网络评论、社会新闻等非结构化数据对系统性金融风险的研究,未能全面利用社会中的各式数据来测度金融风险的传导效应,并且对非结构化数据的处理仍稍显不足。因此,本文以A股上市的20家银行机构为样本数据来源,获取2021年3月1号至2022年2月26号的金融市场数据和互联网文本信息,使用高斯网络模型(Gaussian Network,GN)建立20家上市银行机构的系统性金融风险关联网络,比较不同渠道对系统性金融风险传染效应的影响,并建立外溢性、稳定性和中心性作为系统性重要银行机构评价指标,以评估各家银行在关联网络中所处位置以及风险传染能力,最后为监管部门制定化解金融风险政策提供建议。本文的贡献之处在于,一是通过网络爬虫技术获取银行股吧的股民评论作为非结构化数据,并通过自然语言处理(Natural Language Processing,NLP)提取文本数据的情绪值,扩展了研究系统性金融风险的数据渠道;二是采用机器学习高斯网络模型量化银行机构的关联网络,并使用偏相关系数估计值构建银行机构的系统重要性指标。

\section{模型与方法}
\subsection{基于高斯网络模型的银行体系关联网络构建}
本文为高斯网络模型基础,构建上市银行系统性金融风险关联网络。设$X=(X_1,X_2,\cdots,X_p)\sim N(\mu,\Sigma)$为各银行数据的$p$维随机变量,且服从多元正态分布,一般假设$u=0$。$G=(V,E)$代表无向图,$V=(1,2,\cdots,p)$表示点集合,$E=V\times V$表示边集合,$V$的第$i$行、第$j$列元素$e_{ij}$表示顶点对$i$和$j$是否直接相连,$e_{ij}=1$顶点对表示相连。在除去$X_i$和$X_j$的其他所有随机变量$X_{V-\{i,j}$均确定的条件下,若顶点$i$和顶点$j$不存在相连的边,则$X_i$和$X_j$具备条件独立性,如式(3.13)所示:
\begin{equation}
    e_{ij}=0\Longleftrightarrow X_i\bot X_j|X_{V-\{i,j\}}
\end{equation}
向量$X=(X_1,X_2,\cdots,X_p)$的协方差矩阵用$\Sigma$表示,$\Sigma$的逆矩阵,又名精度矩阵用$\Omega=\Sigma^{-1}$表示,精度矩阵第$i$行第$j$列元素$\sigma_{ij}$存在如下性质:
\begin{equation}
    \rho_{ij|V-\{i,j\}}=0\Longleftrightarrow X_i\bot X_j|X_{V-\{i,j\}}
\end{equation}
其中,
\begin{equation}
    \rho_{ij|V-\{i,j\}}=\frac{-\sigma_{ij}}{\sqrt{\sigma_{ii}\sigma_{jj}}}
\end{equation}
式(3.17)表明,精度矩阵$\Omega$任意元素的值为0,表明顶点对$i$和$j$之间不存在相连的边,等价于随机变量$X_i$和$X_j$具备条件独立性。式(3.17)中的$\rho_{ij|V-\{i,j\}}$代表$X_i$和$X_j$的偏相关系数,即其他随机变量均确定下的相关系数。相较于相关系数,偏相关系数能够更好地衡量银行机构之间的关联方向和程度,经济意义上表示为银行机构$i$对银行机构$j$直接的风险传染程度。本文使用所有随机变量的偏相关系数之和$\sum_{i=1}^p \sum_{j=1}^p \rho_{ij|V-\{i,j\}}$作为衡量整个银行体系系统性金融风险的指标。
给定输入数据,求解高斯网络模型遵循以下步骤:
\begin{enumerate}
    \item 估计精度矩阵,即先估计协方差矩阵,再求其逆矩阵;
    \item 使用嵌套for循环计算偏相关系数矩阵,即可求出高斯网络模型。
\end{enumerate}

对于步骤1,本文采用Friedman et al.(2008)所提出GLasso算法来估计协方差矩阵和精度矩阵,该算法用于求解稀疏逆矩阵,设置L1-范数惩罚项以保证精度矩阵的稀疏性。此外,本文通过K折交叉验证选择最优超参数$\alpha$,$\alpha$为惩罚项系数,数值越大,精度矩阵越稀疏。稀疏逆矩阵的估计方法为以下损失函数的最优化问题:
\begin{equation}
    \hat \Omega=\underset{\Omega}{argmin}\quad [tr(S\Omega)-logdet\Omega+\alpha||\Omega||_1]
\end{equation}
其中,$\Omega$表示待估的精度矩阵,$S$为样本协方差矩阵,$||\Omega||_1$是精度矩阵的非对角元素,即$\sigma_{ij|i\neq j}$。在步骤2中,对估计的精度矩阵$\hat\Omega$根据式(3.18)计算偏相关系数矩阵,即可得出银行机构$i$与$j$的金融风险传染方向和程度。
\subsection{基于BERT-wwm预训练模型的文本数据挖掘}
Bidirection Encoder Representation from Transformers(BERT)属于预训练的语言表征模型,并引入深层双向Transformer编码器来构建模型,BERT-wwm则是在中文语义下对BERT模型的改进。%与LSTM模型相比,Transformer以注意力机制替代了LSTM神经网络所采用RNN结构,能够实现并行计算,提高了运算效率。

\subsubsection{Transformer}
Vaswani et al.(2017)\cite{25}提出了Transformer模型,每个Transformer由6个编码器和6个解码器构成,每个编码器以结合位置编码的词向量为输入,经过注意力层、层标准化(Layer-Normalization)的残差网络和前馈神经网络后得到输出,并传入到下个编码器中。遍历完6个编码器后,将最高层编码器的输出结果分解为向量$K$和$V$,传入解码层,再将输出结果依次传入其余5个解码层,即为完整的Transformer。图\ref{图1}展示了Transformer的完整流程图,图片来自Alammar(2018)\cite{26}。

\begin{figure}[htb]
    \centering
    \includegraphics[scale=0.3]{/Users/singal/Desktop/推免/论文/图片/transformer_resideual_layer_norm_3.png}
    \caption{Transformer流程图}
    \label{图1}
\end{figure}

Transformer的核心为注意力机制,注意力机制的本质思想是为每个单词分配一个权重,该权重通过向量的相似度(内积)来实现,再以该权重来计算所有单词对某个单词的最终输出的影响。注意力机制的实现遵循以下步骤:
\begin{enumerate}
    \item 将每个单词转换为维数相同的词向量$x_i$,词向量的维数属于超参数,论文中给定的维数是512。只有最底层的编码器,即第一个编码器才会接收词向量,而对于其他的编码器,则接收上一层编码器的输出值作为输入值。
    \item 设置三个权重矩阵$W_q,W_k,W_v$,并分别与词向量$x_i$做内积,得到向量$q,k,v$,分别表示“查询”、“键”、“值”;
    \item 再将$q,j$做内积,得到查询与各个键之间的相似度,即某个单词与所有单词(包括本身)的相似度,之后再除以向量的维度(确保方差为1,使梯度向量更稳定),并施加softmax函数,求得各单词被选择的概率;
    \item 最终将单词被选中的概率与向量$v$相乘,即可得到注意力层的输出值:
    \begin{equation}
        Attention(q,k,v)=Softmax\left(\frac{qk^T}{\sqrt {d_k}}\right)v
    \end{equation}
\end{enumerate}

实际计算中是采用基于矩阵的计算方式,对于每个单词$x_i$,均存在与之相对应的$q_i,k_i,v_i$,将所有的$x_i$组合成$X$,则$q_i,k_i,v_i$也组合为$Q,K,V$:
\begin{equation}
    Attention(Q,K,V)=Softmax\left(\frac{QK^T}{\sqrt {d_k}}\right)V
\end{equation}

在原论文中,作者所采用的Multi-head Self-attention由多个$Q,K,V$组成。使用多头注意力机制的优势在于一方面使模型能够关注多个位置的单词信息,另一方面给予注意力层多个“表征子空间”(Representation Subspaces),从而捕捉更丰富的特征信息。此外,作者还在Multi-head Self-attention和前馈神经网络之间加入了Layer Normalization(LN)后的残差网络,并在词向量的编码过程中引入了位置编码,以完善Transformer的整体框架。

\subsubsection{BERT}
BERT由Devlin et al.(2018)\cite{27}提出,是一种深层双向的语言表征模型,只采用了Transformer中的编码器。BERT的输入数据通过Token Embeddings、Segment Embeddings和Position Embeddings三者求和组成。其中,Token Embeddings表示词向量,在中文语义中表示单个词或单个字的语义信息,Segment Embeddings用来区分两个句子。%与Transformer中的位置编码不同,BERT中的Position Embeddings指通过模型学习得到的位置信息,而非通过计算三角函数值确定。

BERT预训练阶段包含两个任务,分别是Masked Language Model(MLM),以及Next Sentence Prediction(NSP)。在MLM任务中,作者随机遮盖(Mask)每个句子中15\%的单词,然后采用非监督学习的方法对被遮盖的单词进行预测。但由于15\%的遮盖率较高,于是作者将被遮盖词中的10\%随机用其他词替代,10\%保持不变,80\%用[MASK]标签遮盖,使用随机替代的原因是:BERT中的双向Transformer编码器必须对每个输入词保持分布式的上下文表征,以避免编码器将[MASK]直接记忆为某个单词。此外,由于随机替代的发生概率只有1.5\%,故不会过多地影响模型对语言理解。BERT预训练阶段的第二个任务是下一句预测(Next Sentence Prediction)如表\ref{表1}所示,用于训练模型对两个句子之间关系的理解能力。

\begin{table}[htb]
    \centering
    \caption{下一句预测的样例}
    \label{表1}
	\begin{tabular*}{\textwidth}{@{}@{\extracolsep{\fill}}cc@{}}
    \toprule
    输入                                                                                                      & 标签       \\ \midrule
    \begin{tabular}[c]{@{}l@{}}{[}CLS{]}我今天只吃了半{[}MASK{]}饭{[}SEP{]}\\ 感觉{[}MASK{]}很饿{[}SEP{]}\end{tabular} & 是上下文连续的  \\
    \begin{tabular}[c]{@{}l@{}}{[}CLS{]}我今天只吃了半{[}MASK{]}饭{[}SEP{]}\\ 今天天气很{[}MASK{]}{[}SEP{]}\end{tabular}  & 不是上下文连续的 \\ \bottomrule
    \end{tabular*}
\end{table}
在该任务中,模型随机地将数据集平均划分,50\%的数据中两个语句是连续的,另50\%的数据中两个句子是不连续的,从而使BERT更好地学习相邻句子之间的相关性,以更好地适用诸如问题回答(Question Answering,QA)自然语言理解(Natural Language Understanding,NLU)等任务。
\subsubsection{BERT-wwm}
%如“这只动物很饿”,“动物”可以表示一个含义,“饿”也能表示一个含义,若仅把“动物”中“动”或“物”遮盖,则明显不符合中文语义。
在英文语义下,一个单词表示一个含义,故在BERT训练阶段的MLM任务中仅需对一个英文单词进行遮盖,但在中文语境下,一个字或一个词均能表示一个含义。故哈工大讯飞联合实验室(2021)\cite{28}发布了基于全词遮罩(Whole Word Masking)技术的中文预训练模型BERT-wwm,主要更改了预训练阶段的样本生成策略,在多项基准测试上获得了进一步性能提升。相较于BERT,BERT-wwm的改进方法为:原有基于WordPiece的分词方式会把一个完整的词切分成若干个子词,在生成训练样本时,这些被分开的子词会随机被mask,在全词Mask中,如果一个完整的词的部分WordPiece子词被mask,则同属该词的其他部分也会被mask,即全词Mask。本文选择BERT-wwm-ext作为预训练模型,该模型使用中文维基百科数据和通用数据,通用数据覆盖百科、新闻、问答等数据,总词数达5.4B,处理后的文本大小约10G。
\subsection{CoVaR与动态风险测度}
Adrian and Brunnermeier(2016)提出通过计算条件风险价值(Conditional Value at Risk,CoVaR)来测度单个金融机构的风险溢出效应,采用该机构分别处于危机和正常状态下的条件风险价值差值$\bigtriangleup CoVaR$来表示。使用$VaR_q^i$表示在$q\%$的概率下,金融机构$i$的最大可能损失,即金融机构$i$损失变量$X_i$的$q\%$分位数:
\begin{equation}
    P(X_i\leq VaR_q^i)=q\%
\end{equation}
与在险价值类似,使用$CoVaR_q^{j|C(X^i)}$表示在金融机构$i$发生事件$C(X^i)$的条件下,金融机构$j$的条件在险价值:
\begin{equation}
    P(X_j\leq CoVaR_q^{j|C(X^i)}|C(X^i))=q\%
\end{equation}
故金融机构$i$对金融机构$j$的风险溢出效应可表示为:
\begin{equation}
    \bigtriangleup CoVaR_q^{j|i}=CoVaR_q^{j|X^i=VaR_q^i}-CoVaR_q^{j|X^i=VaR_{50}^i}
\end{equation}
其中$X^i=VaR_q^i$表示金融机构$i$处于危机状态,$X^i=VaR_{50}^i$表示金融机构$i$处于正常状态。进一步,引入状态变量$M_{t-1}$,将$VaR_{q,t}^i$和$CoVaR_{q,t}^i$表示为$M_{t-1}$的函数,并进行分位数回归:
\begin{align}
    &X_t^i=\alpha_q^i+\beta_q^i M_{t-1}+\epsilon_{q,t}^i\\
    &X_t^{system|i}=\alpha_q^{system|i}+\beta_q^{system|i} M_{t-1}+\gamma_q^{system|i} X_t^i+\epsilon_{q,t}^{system|i}
\end{align}
得到估计值:
\begin{align}
    &VaR_{q,t}^i=\hat \alpha_q^i+\hat\beta_q^i M_{t-1}\\
    &CoVaR_{q,t}^i=\hat\alpha_q^{system|i}+\hat\beta_q^{system|i} M_{t-1}+\hat\gamma_q^{system|i} X_t^i
\end{align}
最终根据式(3.12)计算得到金融机构$i$的动态系统性风险价值$\bigtriangleup CoVaR_{q,t}^i$:
\begin{align}
    \bigtriangleup CoVaR_{q,t}^i&=CoVaR_{q,t}^i-CoVaR_{50,t}^i\notag\\
    &=\hat\beta_q^{system|i}(VaR_{q,t}^i-VaR_{50,t}^i)
\end{align}

\section{数据描述与处理}
本文对A股上市银行按总资产排序,选取前20家银行机构的股价和股吧评论作为样本数据,时间区间为2021年3月1号至2022年2月26号。2021年第三季度财报数据显示,所选取的20家上市银行资产规模占国内全部上市银行资产规模的62.77\%, 负债规模占国内全部上市银行负债规模的62.89\%,并且已覆盖大部分银行业经营业务,故认为该20家上市银行可以作为我国银行体系的代表。表\ref{表2}汇报了截止2021年9月30日20家上市银行的总资产,工商银行资产规模最高,南京银行资产规模最低,前者为后者20.7倍,中国银行与邮储银行的资产差值最大,为14.01万亿元。
\begin{table}[htb]
    \centering
    \caption{20家上市银行机构的总资产}
    \label{表2}
	\begin{tabular*}{\textwidth}{@{}@{\extracolsep{\fill}}cccccccc@{}}
    \toprule
    银行名称 & 总资产   & 银行名称 & 总资产   & 银行名称 & 总资产  & 银行名称 & 总资产  \\ \midrule
    工商银行 & 35.40 & 交通银行 & 11.47 & 民生银行 & 7.02 & 上海银行 & 2.65 \\
    建设银行 & 30.14 & 招商银行 & 8.92  & 光大银行 & 5.69 & 江苏银行 & 2.58 \\
    农业银行 & 28.99 & 浦发银行 & 8.06  & 平安银行 & 4.85 & 浙商银行 & 2.18 \\
    中国银行 & 26.23 & 兴业银行 & 8.50  & 华夏银行 & 3.55 & 宁波银行 & 1.91 \\
    邮储银行 & 12.22 & 中信银行 & 7.89  & 北京银行 & 3.06 & 南京银行 & 1.71 \\ \bottomrule
    \end{tabular*}
\end{table}
\subsection{文本数据}
本文以东方财富网作为文本数据的获取渠道,运用网页爬虫技术爬取各家银行股吧的所有股民评论。由于各家银行的每日评论数不等,为使日期对齐,每家银行的总评论数并不相同。表\ref{表3}汇报了所爬取的银行机构评论页数,兴业银行的评论数最多,累计评论页数为394页,中信银行的评论数最少,累计页数仅为49页,20家银行机构的总评论数为31.3万条。
\begin{table}[htb]
    \centering
    \caption{20家上市银行机构的股吧评论页数}
    \label{表3}
	\begin{tabular*}{\textwidth}{@{}@{\extracolsep{\fill}}cccccccc@{}}
    \toprule
    银行名称 & 帖子页数 & 银行名称 & 帖子页数 & 银行名称 & 帖子页数 & 银行名称 & 帖子页数 \\ \midrule
    工商银行 & 303  & 交通银行 & 175  & 民生银行 & 315  & 上海银行 & 78   \\
    建设银行 & 191  & 招商银行 & 349  & 光大银行 & 194  & 江苏银行 & 168  \\
    农业银行 & 209  & 浦发银行 & 136  & 平安银行 & 369  & 浙商银行 & 96   \\
    中国银行 & 148  & 兴业银行 & 394  & 华夏银行 & 109  & 宁波银行 & 125  \\
    邮储银行 & 220  & 中信银行 & 49   & 北京银行 & 202  & 上海银行 & 87   \\ \bottomrule
    \end{tabular*}
\end{table}

本文采用BERT-wwm预训练模型对每条评论所反映的情绪值进行评估,情绪值越靠近1表明评论所包含的信息越积极,情绪值越接近于0则表示评论越消极。通过BERT-wwm计算出每条评论的情绪值后,再以日期为单位取平均值后即可得到每日投资者情绪值。各家上市银行机构投资者情绪值的均值和标准差如表\ref{表4}所示。所有银行机构的每日投资者情绪平均值均在3.4至3.56附近上下波动,平安银行的情绪均值最高,为0.36,上海银行的情绪均值最低,为0.34。银行机构的情绪值波动幅度也有所不同,上海银行的情绪值标准差最大,兴业银行的情绪标准差最低,上海银行同时兼具最低情绪均值和最高情绪波动,又因为情绪值越接近0表示文本信息越消极,故上海银行的文本数据反映出其经营状况在所选银行机构中处于劣势。
\begin{table}[htb]
    \caption{20家上市银行机构的投资者情绪平均值和标准差}
    \label{表4}
    \begin{tabular*}{\textwidth}{@{}@{\extracolsep{\fill}}cccccc@{}}
    \toprule
    银行名称 & 情绪平均值    & 情绪值标准差   & 银行名称 & 情绪平均值    & 情绪值标准差   \\ \midrule
    工商银行 & 0.3446 & 0.0106 & 民生银行 & 0.3513 & 0.0068 \\
    建设银行 & 0.3465 & 0.0101 & 光大银行 & 0.3520 & 0.0102 \\
    农业银行 & 0.3507 & 0.0090 & 平安银行 & 0.3556 & 0.0099 \\
    中国银行 & 0.3466 & 0.0118 & 华夏银行 & 0.3473 & 0.0119 \\
    邮储银行 & 0.3519 & 0.0130 & 北京银行 & 0.3459 & 0.0082 \\
    交通银行 & 0.3500 & 0.0100 & 上海银行 & 0.3443 & 0.0137 \\
    招商银行 & 0.3499 & 0.0094 & 江苏银行 & 0.3483 & 0.0129 \\
    浦发银行 & 0.3481 & 0.0113 & 浙商银行 & 0.3517 & 0.0121 \\
    兴业银行 & 0.3530 & 0.0087 & 宁波银行 & 0.3511 & 0.0137 \\
    中信银行 & 0.3421 & 0.0150 & 南京银行 & 0.3505 & 0.0132 \\ \bottomrule
    \end{tabular*}
\end{table}

\subsection{市场数据}
市场数据指各家上市银行在A股市场的每日收益率,由于对数收益率具备一些优良性质,如将价格序列转换为平稳序列,同时数值上与真实收益率近似相等,故本文使用银行的每日对数收盘价的一阶差分来衡量每日收益率:
\begin{equation}
    R_t=log(P_t/P_{t-1})
\end{equation}
式(4.1)中$t$代表某日,$t-1$代表前一日,$P$表示价格。

\subsection{动态风险数据}
本文选取20家上市银行机构每日交易日的收盘价相对于上一交易日收盘价的涨跌幅,并取相反数作为各家银行损失率的代理变量,再将各家银行损失率以市值为权重加权求和,得到银行系统损失率的代理变量。损失率的描述性统计如表\ref{表5}所示,在20家上市银行机构中,民生银行损失率均值最高,为0.11,江苏银行的损失率最低且为负,即实现了盈利。就波动幅度而言,宁波银行、平安银行的标准差最高,四分位数相隔距离也较远,表明该两家银行的损失具备较高的不稳定性。此外,银行系统损失率的均值为0,标准差、四分位数、最值均接近于0,表明我国银行业整体的损失水平较低。

\setlength{\tabcolsep}{9.7pt}
\begin{longtable}{cccccccc}
    \caption{银行机构损失率的描述性统计}
    \label{表5}\\
    \toprule
         & 平均值   & 标准差  & 最小值    & 25\% 分位数 & 50\%分位数 & 75\% 分位数 & 最大值  \\ \midrule
    \endhead
    \bottomrule
    \endfoot
    \endlastfoot
    工商银行 & 0.04  & 0.83 & -3.02  & -0.42    & 0.19    & 0.43     & 4.81 \\
    建设银行 & 0.06  & 1.10 & -6.24  & -0.50    & 0.17    & 0.67     & 3.81 \\
    农业银行 & 0.02  & 0.74 & -2.18  & -0.34    & 0.00    & 0.34     & 5.81 \\
    中国银行 & 0.01  & 0.63 & -1.62  & -0.33    & 0.00    & 0.33     & 5.67 \\
    邮储银行 & -0.00 & 1.66 & -5.15  & -0.89    & 0.19    & 1.16     & 6.78 \\
    交通银行 & -0.04 & 0.92 & -3.86  & -0.43    & 0.00    & 0.43     & 7.26 \\
    招商银行 & -0.01 & 2.05 & -7.09  & -1.13    & 0.18    & 1.09     & 6.15 \\
    浦发银行 & 0.09  & 1.06 & -5.06  & -0.38    & 0.09    & 0.58     & 4.65 \\
    兴业银行 & 0.04  & 2.05 & -6.88  & -1.19    & 0.16    & 1.31     & 5.84 \\
    中信银行 & 0.04  & 0.95 & -2.67  & -0.43    & 0.00    & 0.55     & 7.46 \\
    民生银行 & 0.11  & 0.78 & -1.79  & -0.25    & 0.00    & 0.45     & 4.17 \\
    光大银行 & 0.08  & 0.97 & -2.99  & -0.30    & 0.25    & 0.59     & 6.03 \\
    平安银行 & 0.08  & 2.27 & -8.26  & -1.02    & 0.28    & 1.29     & 6.47 \\
    华夏银行 & 0.03  & 0.81 & -2.98  & -0.32    & 0.00    & 0.46     & 5.37 \\
    北京银行 & 0.01  & 0.75 & -2.29  & -0.41    & 0.00    & 0.42     & 5.80 \\
    上海银行 & 0.07  & 0.92 & -2.67  & -0.46    & 0.13    & 0.53     & 4.17 \\
    江苏银行 & -0.09 & 1.96 & -6.97  & -1.12    & 0.00    & 1.01     & 6.41 \\
    浙商银行 & 0.06  & 0.72 & -2.60  & -0.28    & 0.00    & 0.29     & 5.97 \\
    宁波银行 & -0.02 & 2.44 & -10.00 & -1.23    & 0.06    & 1.48     & 5.81 \\
    南京银行 & -0.07 & 2.04 & -10.03 & -1.10    & 0.11    & 1.13     & 7.47 \\
    银行系统 & 0.00  & 0.05 & -0.21  & -0.03    & 0.01    & 0.03     & 0.13 \\ \bottomrule
\end{longtable}

为计算动态CoVaR,本文选取状态变量,包括:
\begin{enumerate}
    \item 国债到期收益率的变化,采用3月期国债到期收益率的一阶差分来表示;
    \item 长短期利差,使用10年期国债到期收益率和3月期国债到期收益率之差作为代理变量;
    \item 流动性利差,采用3月期上海银行间同业拆放利率(SHIBOR)和3月期国债到期收益率之差作为代理变量;
    \item 信用利差的变化,信用利差用10年期AAA级企业债到期收益率和10年期国债到期收益率的差来表示;
    \item 市场收益率,用上证综指每日涨跌幅的对数一阶差分来表示;
    \item 波动率,将上证综指日度涨跌幅的22天滚动标准差作为代理变量。
\end{enumerate}

将各家银行机构的损失率分布作为式(3.11)的被解释变量$X_t^i$,状态变量$M_{t-1}$作为解释变量,分别估计95\%和50\%分位数回归模型,再将银行系统损失率$X_t^{system|i}$作为被解释变量,$M_{t-1}$和$X_t^i$作为解释变量估计95\%分位数回归模型。将式(3.13)估计得到$VaR_{95,t}^i$和$VaR_{50,t}^i$预测值和式(3.14)估计得到的$\hat\beta_q^{system|i}$代入式(3.15)中,即可得到20家上市银行机构的动态$\bigtriangleup CoVaR_{95,t}^i$,结果如图\ref{图3}所示。
\begin{figure}[htb]
    \centering
    \includegraphics[scale=0.34]{/Users/singal/Desktop/推免/论文/图片/CoVaR趋势图.png}
    \caption{银行机构的动态风险数据趋势图}
    \label{图3}
\end{figure}

由图\ref{图3}可得,20家银行的系统性风险总体呈现出“均值回复”的特性,在2021年6月,工商银行、农业银行、中国银行等多家银行机构的系统性风险大幅上升,其余时间段无明显波动。此外,国有大型商业银行的波动率明显低于民营银行,反映出国有大型商业银行在促进银行系统稳定方面具有重要作用。在2022年2月末,建设银行、农业银行、中国银行等均出现了系统性风险波动加剧的特征,预示未来系统性金融风险防控仍是重中之重。

\section{基于金融网络的系统性金融风险传染分析}
\subsection{系统性金融风险传染网络图}
本文以20家上市银行机构的市场数据、投资者情绪数据以及动态风险数据作为高斯网络模型的输入数据,通过求解精度矩阵得到银行机构间的偏相关系数矩阵,并以各银行机构作为节点,以偏相关系数的数值大小作为边的宽度,绘制出三种不同输入数据的关联网络图,如图\ref{图4}、图\ref{图5}和图\ref{图6}所示。

图\ref{图4}展示了基于投资者情绪数据估计的高斯网络模型,实线越粗表示银行机构间的偏相关系数越大,即风险传染性越强。总体而言,由于该图的密度为0.4349,故投资者情绪在金融风险传染渠道中所占比重较高,银行机构的关联性较强。具体到个别银行,从边的粗细可得,工商银行和建设银行的偏相关系数值最大,表明两家银行间基于投资者情绪的金融风险传染效应较强,此外,工商银行与平安银行、江苏银行与华夏银行等银行机构间也具备较强的风险传染效应。
\begin{figure}[htb]
    \centering
    \includegraphics[scale=0.86]{/Users/singal/Desktop/推免/论文/图片/cor_cv_text.png}
    \caption{投资者情绪的关联网络图}
    \label{图4}
\end{figure}

基于市场数据所估计的高斯网络模型如图\ref{图5}所示,直观上看,相较于投资者情绪值,市场数据反映出上市银行间具备更高的偏自相关系数,即金融风险的传导效应较强。但该图的密度为0.4404,仅略高出了图\ref{图4},这是由于指标的选择存在弊端。图密度只考虑了节点对之间是否存在边,而忽视了边的权重大小。具体到个别银行,招商银行与宁波银行、招商银行与平安银行的偏相关系数最大,体现出较强的金融风险传染效应,同时图\ref{图3}中所反映的工商银行与建设银行的风险传染效应在图\ref{图4}中并无明显变化,可见招商银行对基于股价收益率的金融风险传染效应更为敏感。

\begin{figure}[htb]
    \centering
    \includegraphics[scale=0.86]{/Users/singal/Desktop/推免/论文/图片/cor_cv_market.png}
    \caption{市场数据的关联网络图}
    \label{图5}
\end{figure}

图\ref{图6}为利用动态风险数据所构建的高斯网络模型,该图的密度为0.3850,比图\ref{图4}和图\ref{图5}更为稀疏,但个别实线明显粗于前两幅图,同样反映出图密度指标的局限性。工商银行与上海银行、工商银行与建设银行的偏相关系数最大,反映尾部风险对三家银行具备更强的关联性,这与图3的结论部分一致。此外,兴业银行与平安银行、浦发银行与广大银行业具备较强的关联性。
\begin{figure}[htb]
    \centering
    \includegraphics[scale=0.86]{/Users/singal/Desktop/推免/论文/图片/cor_cv_covar.png}
    \caption{动态风险数据的关联网络图}
    \label{图6}
\end{figure}

考虑图密度这一指标的缺点,本文将高斯网络模型的偏相关系数绝对值加总求和,并计算网络图片的平均聚集系数和平均路径长度,作为系统性金融风险量化指标,以更准确地衡量三种金融风险传染渠道。计算公式如下:
\begin{equation}
    \text{偏相关系数之和}=\sum_{i=1}^{N}\sum_{j\neq i}^{N} |\rho_{ij}|
\end{equation}
其中,$p$为银行机构的个数,在本文中$N=20$。由于本文构建的银行机构关联网络为有权无向图,故采取Dijkstra算法计算平均路径长度:
\begin{equation}
    \text{平均路径长度}\quad L=\frac{1}{N(N-1/2)}\sum_{i\geq j}d_{ij}
\end{equation}
式(5.2)中$d_{ij}$表示网络中两个节点$i$和$j$之间边的权值之和最小的路径上边的数目,即测地距离。
\begin{equation}
    \text{节点的聚类系数}\quad C_i=\frac{E_i}{k_i(k_i-1)/2}
\end{equation}
\begin{equation}
    \text{网络的平均聚类系数}\quad C=\frac{1}{N}\sum_{i=1}^N C_i
\end{equation}
式(5.3)中$E_i$是节点$i$的$k_i$个邻节点之间实际存在的边数,对网络中所有节点的聚类系数计算平均值,即得网络的平均聚类系数。
根据式(5.1)至(5.4)所计算出不同渠道的系统性金融风险衡量指标如表\ref{表6}所示,为便于对比,图密度也列于表\ref{表6}中。

\begin{table}[h]
    \centering
    \caption{系统性金融风险衡量指标对比}
    \label{表6}
    \begin{tabular*}{\textwidth}{@{}@{\extracolsep{\fill}}cccc@{}}
    \toprule
    指标       & 投资者情绪数据 & 市场数据   & 动态风险数据 \\ \midrule
    图密度      & 0.4349  & 0.4404 & 0.3850  \\
    平均聚类系数   & 0.1114  & 0.1667 & 0.1202 \\
    平均路径长度   & 0.0368  & 0.0594 & 0.0372 \\
    $\sum_{i=1}^{p}\sum_{j\neq i}^{p} |\rho_{ij}|$  & 9.67    & 13.63  & 15.71  \\ \bottomrule
    \end{tabular*}
\end{table}
偏相关系数之和与图密度呈现出相反的变化趋势,基于投资者情绪值所估计的高斯网络模型反映出较高的图密度,但却具备最低的偏相关系数之和,而动态风险数据则反映出最高的偏相关系数之和和最低的图密度,这与图\ref{图5}中对图密度缺点的分析结果一致。就平均聚类系数和平均路径长度而言,基于市场数据的银行机构关联网络反映出最高的关联水平,表明系统性金融风险通过该渠道的传染效应最强,动态风险数据次之,投资者情绪数据反映的系统性金融风险强度最弱。综合考虑四种指标的计算结果以及网络图\ref{图4}、\ref{图5}、\ref{图6}中的线条粗细,可认为基于市场波动和尾部风险渠道的系统性金融风险传染效应最强。

由于三种关联网络均具备较高的平均聚集系数和相对较低的平均路径长度,因此我国上市银行机构关联网络可能具备“小世界现象”。为验证猜想,本文根据Humphries(2006,2008)所提出的$\sigma$系数,以及Telesford(2011)提出的$\Omega$系数来评估“小世界现象”存在的可能性。$\sigma$系数和$\Omega$系数的计算方式如下:
\begin{align}
    \sigma&=\frac{C/C_r}{L/L_r}\\
    \Omega&=\frac{L_r}{L}-\frac{C}{C_l}
\end{align}
其中,$C$和$L$分别是银行机构关联网络的平均聚集系数和最短路径长度,$C_r$和$L_r$分别是随机图的平均聚类系数和最短路径长度,$C_l$是lattice图的平均聚类系数,该随机图和lattice图与银行机构关联网络具备相同节点数目。若$\sigma>1$或$\Omega$接近于0,可认为银行机构关联网络存在“小世界现象”。结果如表\ref{表7}所示。
\begin{table}[h]
    \centering
    \caption{“小世界现象”的量化指标}
    \label{表7}
    \begin{tabular*}{\textwidth}{@{}@{\extracolsep{\fill}}cccc@{}}
    \toprule
    指标       & 投资者情绪数据 & 市场数据   & 动态风险数据 \\ \midrule
    $\sigma$系数 & 0.9920  & 0.9838 & / \\
    $\Omega$系数 & 0.0310  & 0.0196 & / \\ \bottomrule
    \end{tabular*}
\end{table}

基于动态风险数据所估计的关联网络无法直接计算平均路径长度,因此无法得出$\sigma$系数和$\Omega$系数。由投资者情绪和市场数据所估计关联网络的$\sigma$系数分别为0.9920和0.9839,小于但接近于1,$\Omega$系数分别为0.0310和0.0196,极为接近0,故认为银行机构关联网络具备“小世界现象”。

\subsection{系统性金融风险传染热力图}
本文对三种数据所计算出的银行机构间偏相关系数绘制热力图,以更详细地分析银行间金融风险的传染效应,热力图如图\ref{图7}所示。为方便对比,设置银行与其自身的偏相关系数为1,图右边为偏相关系数的“色柱”,颜色越深代表偏相关系数的绝对值越大。图中第一幅图为基于投资者情绪的系统性金融风险传染热力图,工商银行和建设银行、工商银行与平安银行的负向关联程度较高,偏相关系数分别为-0.2016和-0.1751,与网络图的分析一致,表明国有大型银行之间、国有大型银行与股份制商业银行间的关联性较强。此外,宁波银行与交通银行、宁波银行与江苏银行的负向关联程度也较高,偏相关系数分别为-0.1438和-0.1409,可见热力图所反映的信息是对网络图的补充。图中第二幅图为使用市场数据,即股价收益率的系统性金融风险传染热力图,相较于投资者情绪,市场数据的热力图所反映的偏相关系数总体上更接近于0,即风险的稳定性不如投资者情绪明显。工商银行与建设银行仍然具有较强的关联性,招商银行与平安银行、招商银行与宁波银行的偏相关系数最高,分别为-0.3140和-0.2471,与网络图的分析相一致,兴业银行与平安银行的偏相关系数较低,为-0.2274,可见股份制商业银行间同样具备较高的金融风险传染能力。基于动态风险数据的系统性金融风险传染热力图为图中第三幅图,反映出利用动态CoVaR所计算出的风险稳定性大致不及投资者情绪和市场数据。具体到个别银行,工商银行与建设银行、工商银行与上海银行的偏相关系数最低,分别为-0.3667和-0.3155,表现工商银行与另外两家银行间存在较强的风险稳定性。一些股份制商业银行,如光大银行与浦发银行、平安银行与兴业银行间也存在较强的负向关联性。

\begin{figure}[htb]
    \centering
    \includegraphics[scale=0.49]{/Users/singal/Desktop/推免/论文/图片/heatmap.png}
    \caption{系统性金融风险传染热力图}
    \label{图7}
\end{figure}
\subsection{银行机构金融风险传染能力}
为分析网络图和热力图中的关键银行节点,本文以每家银行为单位,计算与其他银行的偏相关系数之和,再将三种数据的计算所得加总,作为衡量该家银行系统性金融风险传染能力的指标。20家上市银行机构的风险传染能力如表\ref{表8}所示。

\setlength{\tabcolsep}{5pt}
\begin{table}[htb]
    \centering
    \caption{银行机构系统性金融风险传染能力}
    \label{表8}
    \begin{tabular}{@{}cccccccccccc@{}}
    \toprule
    排序 & 银行机构 & 传染能力 & 排序 & 银行机构 & 传染能力 & 排序 & 银行机构 & 传染能力 & 排序 & 银行机构 & 传染能力 \\ \midrule
    1  & 南京银行 & 6.30 & 6  & 光大银行 & 5.38 & 11 & 宁波银行 & 4.83 & 16 & 邮储银行 & 4.42 \\
    2  & 兴业银行 & 6.10 & 7  & 江苏银行 & 5.37 & 12 & 浦发银行 & 4.73 & 17 & 交通银行 & 4.23 \\
    3  & 上海银行 & 5.49 & 8  & 建设银行 & 5.27 & 13 & 民生银行 & 4.55 & 18 & 中国银行 & 4.21 \\
    4  & 平安银行 & 5.45 & 9  & 招商银行 & 5.14 & 14 & 浙商银行 & 4.55 & 19 & 北京银行 & 3.91 \\
    5  & 工商银行 & 5.40 & 10 & 农业银行 & 4.94 & 15 & 中信银行 & 4.45 & 20 & 华夏银行 & 3.76 \\ \bottomrule
    \end{tabular}
\end{table}

综合三种数据的系统性金融风险传染网络图、热力图以及表\ref{表8}所反映的银行机构金融风险传染能力,本文得出银行机构关联网络的相关结论:
\begin{enumerate}
    \item 我国上市银行机构的关联网络结构具备“无标度特性”,即少数关键银行机构与其他银行机构具备较强的关联性,而其余大部分银行机构之间的关联性较弱,如工商银行、招商银行、宁波银行等均与其他多家银行具备较高的关联程度,而浙商银行、北京银行、华夏银行等银行机构与其他银行的关联性不高。“无标度特性”使我国银行系统具备“稳健且脆弱”的倾向,由于大部分银行机构的连通性较低,故随机冲击影响低联通节点的概率较高,风险发生概率较低。若随机冲击小概率地影响到了某个关键银行机构,则该银行机构所具备的高关联性将会对银行系统产生严重破坏。
    \item 通过计算$\sigma$和$\Omega$系数认为我国银行机构关联网络存在“小世界现象”,只需少量中间银行即可连通任意两个银行机构,金融风险在银行机构关联网络中的传染路径较短、传染速度较快;
    \item 综合比较三种关联网络的图密度、平均聚类系数、平均路径长度和偏相关系数之和,本文认为基于尾部风险和市场波动的系统性金融风险传染效应较强,而基于投资者情绪的系统性金融风险传染效应最弱;
    \item 如表\ref{表8}所示,系统性金融风险传染能力排名前5位中,有4家为股份制商业银行。后危机时代,部分股份制商业银行的高财务杠杆率和业务过度表外化致使我国影子银行业务迅速扩张,这将导致我国金融系统的关联程度大幅提高,带来系统性金融风险爆发的隐患,因此监管部门一方面应重视“太大而不能倒”的国有大型商业银行的监管力度,另一方面应限制“太关联而不能倒”的股份制商业银行影子业务的快速扩张。
\end{enumerate}

\section{中国银行体系系统重要性机构识别}
本文分别从外溢性、稳定性以及多种中心性指标构建系统重要性银行机构的识别指标。首先根据高斯网络模型所得出的偏相关系数矩阵估计值,以银行资产规模为权重,计算各家银行系统性金融风险的外溢性和稳定性指标。风险外溢性是指银行A与银行B存在正的相关系数,A风险的增加使B的风险同时提高,即A风险“溢出”给了B;稳定性是指银行A与银行B呈负相关关系,A风险增加,B风险减少,使系统性金融风险相对降低。外溢性、稳定性和系统性重要指标的计算公式如下:
\begin{align}
    SP_i&=\sum_{j\in neighbor(i)}^p \rho_{ij}^{+}\cdot w_{ij}\\
    ST_i&=\sum_{j\in neighbor(i)}^p |\rho_{ij}^{-}|\cdot w_{ij}\\
    w_{ij}&=\frac{Asset_i+Asset_j}{\sum_k^p Asset_k}\\
    SI_i&=SP_i+ST_i
\end{align}
其中,$SP_i,ST_i,SI_i$分别表示银行机构$i$的外溢性、稳定性和系统重要性指标,$p$表示银行机构的数量,本文中$p=20$,$\rho_{ij}^{+}$表示为正值的相关系数,$\rho_{ij}^{-}$表示为负值的相关系数,$w_{ij}$表示银行机构$i$和$j$资产之和占全部银行资产之和的比重。

表\ref{表9}汇报了20家上市银行机构的外溢性、稳定性和系统性指标的计算结果,已按降序排序,由于四舍五入,部分数据显示为0。外溢性指标的前5家银行机构中,国有大型商业银行占据4家,分别为交通银行、农业银行、工商银行和建设银行,表明资产规模大、社会影响力强的银行机构往往具备较强的外溢性。南京银行是唯一一家外溢性指标位于前5名的股份制商业银行,反映出南京银行的风险传染效应强于其他股份制商业银行。外溢性最低的则是招商银行。浦发银行、华夏银行等股份制商业银行,邮储银行和中国银行作为国有商业银行,也具备较低的外溢性,表明其风险传染能力较弱。稳定性指标的前5位同样包含3家国有大型商业银行,并且中国银行和邮储银行的次序有所上升,分别位列第7和第8,表明国有大型商业银行在系统性金融风险传染中占据主要地位。此外,在外溢性指标中排名第1的南京银行,在稳定性指标中排名第12,反映出南京银行的金融风险主要通过正向关系向其他金融机构传导。系统性重要指标中,前5名的银行机构未发生更迭,中国银行和邮储银行的次序有所下降,华夏银行和北京银行在系统性金融风险传导中依然占据较低的比重。

综合考察三种度量指标,稳定性指标在系统重要性指标中占据较大比重,平均比重为90.73\%,表明我国银行机构主要是通过风险稳定性来影响其他银行机构的,即银行机构风险的减少将会使其他银行风险增加,二者呈反方向变动。此外,国有大型商业银行在系统性金融风险传染过程中起到重要作用,诸如南京银行等股份制商业银行在业务运营过程中也具备较大的风险外溢性。

\setlength{\tabcolsep}{11.2pt}
\begin{longtable}[c]{@{}ccccccccc@{}}
    \caption{银行机构外溢性、稳定性和系统重要性指标结果}
    \label{表9}\\
    \toprule
    排序 & 银行名称 & 外溢性   & 排序 & 银行名称 & 稳定性   & 排序 & 银行名称 & 系统重要性 \\ \midrule
    \endhead
    %
    \bottomrule
    \endfoot
    %
    \endlastfoot
    %
    1  & 交通银行 & 0.063 & 1  & 工商银行 & 0.572 & 1  & 工商银行 & 0.607 \\
    2  & 南京银行 & 0.057 & 2  & 建设银行 & 0.529 & 2  & 建设银行 & 0.552 \\
    3  & 农业银行 & 0.045 & 3  & 招商银行 & 0.369 & 3  & 招商银行 & 0.369 \\
    4  & 工商银行 & 0.034 & 4  & 农业银行 & 0.305 & 4  & 农业银行 & 0.350 \\
    5  & 建设银行 & 0.023 & 5  & 兴业银行 & 0.243 & 5  & 兴业银行 & 0.258 \\
    6  & 上海银行 & 0.023 & 6  & 平安银行 & 0.197 & 6  & 平安银行 & 0.210 \\
    7  & 江苏银行 & 0.023 & 7  & 中国银行 & 0.177 & 7  & 南京银行 & 0.189 \\
    8  & 宁波银行 & 0.016 & 8  & 邮储银行 & 0.165 & 8  & 中国银行 & 0.181 \\
    9  & 兴业银行 & 0.015 & 9  & 光大银行 & 0.159 & 9  & 邮储银行 & 0.165 \\
    10 & 平安银行 & 0.013 & 10 & 宁波银行 & 0.141 & 10 & 光大银行 & 0.164 \\
    11 & 中信银行 & 0.006 & 11 & 上海银行 & 0.132 & 11 & 宁波银行 & 0.157 \\
    12 & 光大银行 & 0.006 & 12 & 南京银行 & 0.132 & 12 & 上海银行 & 0.155 \\
    13 & 民生银行 & 0.006 & 13 & 浦发银行 & 0.117 & 13 & 交通银行 & 0.128 \\
    14 & 中国银行 & 0.004 & 14 & 浙商银行 & 0.107 & 14 & 浦发银行 & 0.117 \\
    15 & 浙商银行 & 0.002 & 15 & 中信银行 & 0.080 & 15 & 浙商银行 & 0.110 \\
    16 & 北京银行 & 0.001 & 16 & 民生银行 & 0.080 & 16 & 江苏银行 & 0.095 \\
    17 & 邮储银行 & 0.000 & 17 & 江苏银行 & 0.072 & 17 & 中信银行 & 0.086 \\
    18 & 招商银行 & 0.000 & 18 & 交通银行 & 0.065 & 18 & 民生银行 & 0.086 \\
    19 & 浦发银行 & 0.000 & 19 & 北京银行 & 0.039 & 19 & 北京银行 & 0.039 \\
    20 & 华夏银行 & 0.000 & 20 & 华夏银行 & 0.034 & 20 & 华夏银行 & 0.034 \\ \bottomrule
\end{longtable}

其次,分别计算三种关联网络的度中心性(Degree centrality,DC)和特征向量中心性(Eigenvector centrality,EC)作为衡量银行机构系统性重要性的补充。若网络共包含$N$个节点,则度为$k_i$的节点的归一化度中心性(DC)值定义为:
\begin{equation}
    DC_i=\frac{k_i}{N-1}
\end{equation}
特征向量中心性(EC)的基本思想是:一个节点重要性既取决于节点的度,也取决于邻居节点的重要性。参照的描述,记银行机构$i$的重要性度量值为$x_i$,则:
\begin{equation}
    x_i=c\sum_{j=1}^N a_{ij}x_j
\end{equation}
其中$c$为一常数,$\bm{A}=(a_{ij})$是银行机构关联网络的邻接矩阵。记$\bm{x}=[x_1,x_2,\cdots,x_N]^T$,则可化为矩阵形式:
\begin{equation}
    \bm{x}=c\bm{A}\bm{x}
\end{equation}
观察上式可得,$\bm{x}$为矩阵$\bm{A}$的特征向量,$c^{-1}$为对应的特征值。采用如下迭代算法即可求出向量$\bm{x}$为邻接矩阵的最大特征值所对应的特征向量。
\begin{equation}
    \bm{x}(k)=c\bm{A}\bm{x}(k-1),\quad k=1,2,\cdots
\end{equation}
根据式(5.11)和式(5.14)计算出三种关联网络的银行机构度中心性(DC)和特征向量中心性(EC),并求和得到总计列,如表\ref{表10}所示,由于EC所反映的信息更全面,故按照总计列的EC降序排序。

\setlength{\tabcolsep}{12.7pt}
\begin{longtable}[c]{@{}ccccccccc@{}}
    \caption{银行机构中心性的评估指标结果}
    \label{表10}\\
    \toprule
    & \multicolumn{2}{c}{投资者情绪数据} & \multicolumn{2}{c}{市场数据} & \multicolumn{2}{c}{动态风险数据} & \multicolumn{2}{c}{总计}\\
    银行名称 & EC & DC & EC & DC & EC & DC & EC & DC \\\midrule
    \endhead
    %
    \bottomrule
    \endfoot
    %
    \endlastfoot
    %
    南京银行 & 0.2450 & 0.5263 & 0.3337 & 0.8947 & 0.3124 & 0.6842 & 0.8911 & 2.1053 \\
    光大银行 & 0.3077 & 0.6842 & 0.2340 & 0.4737 & 0.3420 & 0.7895 & 0.8838 & 1.9474 \\
    江苏银行 & 0.3273 & 0.7368 & 0.2851 & 0.6842 & 0.1936 & 0.3684 & 0.8061 & 1.7895 \\
    兴业银行 & 0.2199 & 0.4737 & 0.3453 & 0.9474 & 0.2049 & 0.4211 & 0.7701 & 1.8421 \\
    上海银行 & 0.2655 & 0.5789 & 0.1907 & 0.3684 & 0.3098 & 0.7368 & 0.7660 & 1.6842 \\
    建设银行 & 0.2117 & 0.4737 & 0.2534 & 0.5789 & 0.2237 & 0.4737 & 0.6889 & 1.5263 \\
    平安银行 & 0.2027 & 0.4211 & 0.2637 & 0.6842 & 0.2037 & 0.4211 & 0.6702 & 1.5263 \\
    工商银行 & 0.2507 & 0.5789 & 0.2122 & 0.4211 & 0.1861 & 0.3684 & 0.6490 & 1.3684 \\
    招商银行 & 0.2871 & 0.5789 & 0.3232 & 0.8421 & 0.0000 & 0.0000 & 0.6103 & 1.4211 \\
    农业银行 & 0.1636 & 0.3158 & 0.1373 & 0.2632 & 0.3035 & 0.6842 & 0.6045 & 1.2632 \\
    民生银行 & 0.2162 & 0.4211 & 0.1174 & 0.2105 & 0.2709 & 0.5789 & 0.6045 & 1.2105 \\
    宁波银行 & 0.2513 & 0.5789 & 0.2828 & 0.6316 & 0.0617 & 0.1053 & 0.5958 & 1.3158 \\
    浙商银行 & 0.1861 & 0.4211 & 0.0798 & 0.1579 & 0.3294 & 0.7895 & 0.5953 & 1.3684 \\
    中国银行 & 0.2930 & 0.6316 & 0.0630 & 0.1053 & 0.2208 & 0.4211 & 0.5768 & 1.1579 \\
    中信银行 & 0.1976 & 0.4211 & 0.1645 & 0.3158 & 0.2019 & 0.3684 & 0.5640 & 1.1053 \\
    浦发银行 & 0.1202 & 0.2632 & 0.1626 & 0.3158 & 0.2334 & 0.4737 & 0.5162 & 1.0526 \\
    邮储银行 & 0.1868 & 0.4211 & 0.2740 & 0.6842 & 0.0000 & 0.0000 & 0.4608 & 1.1053 \\
    交通银行 & 0.1361 & 0.2632 & 0.1128 & 0.2105 & 0.1983 & 0.3684 & 0.4473 & 0.8421 \\
    北京银行 & 0.0898 & 0.2105 & 0.1438 & 0.2632 & 0.1502 & 0.2632 & 0.3838 & 0.7368 \\
    华夏银行 & 0.1265 & 0.2632 & 0.1700 & 0.3158 & 0.0000 & 0.0000 & 0.2965 & 0.5789 \\ \bottomrule
\end{longtable}
由表\ref{表10}可得,南京银行、光大银行、江苏银行等五家股份制商业银行的中心性指标占据前列,高于建设银行(第6)、工商银行(第8)等国有大型商业银行。外溢性、稳定性和系统重要性指标结果反映出“资产规模”是影响银行机构重要性的主要因素,而中心性指标反映出“关联性”更强的股份制商业银行具备更强的金融风险传染能力,故系统重要性指标和中心性指标是对银行机构风险传染能力的两种不同反映。一方面,资产规模越大的银行,通常与其他金融机构开展更多同业业务,形成共同风险敞口,从而具备较高的系统关联性;另一方面,关联程度越高的银行往往业务种类繁多,资产业务和负债业务能够涉及更多细分市场,从而提高资产规模和利润率。因此,“太大而不能倒”与“太关联而不能倒”是相关问题,而不是独立问题。
\section{结论与建议}

基于系统性金融风险传染的视角,本文以投资者情绪、市场数据和动态风险数据为研究渠道,运用高斯网络模型建立了20家上市银行机构的系统性金融风险关联网络,并对样本银行的系统性金融风险传染方向、程度和渠道进行了实证分析,最后构建外溢性和稳定性的评价指标,以评估金融风险传染网络中各银行的系统重要程度。

研究结果表明,首先,系统性金融风险传染效应的强弱与传染渠道具有密切联系,基于尾部风险和市场波动的系统性金融风险传染效应较强,而基于投资者情绪数据的系统性金融风险传染效应最弱,由于尾部风险和市场波动是通过股票的涨跌幅计算得出,故反映出当前我国的系统性金融风险传染渠道主要是股价这一媒介,而互联网文本数据,即股民评论所起作用较弱。其次,我国上市银行机构关联网络呈现“无标度特征”,即关联网络分布严重不均匀,少数银行机构对网络的运行起主导作用,银行机构关联网络具备“稳健且脆弱”的倾向,即随机冲击影响低联通节点的概率较高,但违约概率较低,但若随机冲击影响了某个高联通节点,则银行系统整体将发生严重的系统性金融风险。综合系统性金融风险的三种传染渠道来看,工商银行、招商银行、宁波银行等均与其他多家银行具备较强的关联性,而浙商银行、华夏银行、北京银行等银行机构与其他银行的关联性不强。最后,结合各家银行机构的风险传染能力以及外溢性、稳定性、系统重要性评价指标可得,由于国有大型商业银行是央行通过货币政策调控宏观经济并疏散风险的重要渠道,故以工、农、建为代表的国有大型商业银行在系统性金融风险传染网络中占据核心地位;同时,度中心性和特征向量中心性评价指标反映诸如南京银行、光大银行、江苏银行、兴业银行等股份制商业银行的系统性金融风险传染能力最强,因此资产规模与关联性仍是我国银行系统风险传染能力的主要影响因素。

本文的建议包括下述三点:第一,从多渠道对系统性金融风险传染效应进行监管,在加强对金融市场审查的同时,也应重视对互联网文本数据等新型风险传染渠道的监测,以把握信息来源的全面性,有利于从宏观审慎的角度维护金融业的安全与稳定。第二,根据我国系统性金融风险传染网络图谱所反映出的“无标度特性”和“小世界特性”,监管部门应依据系统重要性指标和中心性指标识别出中心银行机构,对不同类型的金融机构应实行差异化监管,重点关注中心银行机构,进而降低银行机构个体风险在金融网络中的传染速度和范围,实现降低系统性金融风险的目的;第三,度中心性和特征向量中心性评价指标表明股份制商业银行的风险传染能力强于国有大型商业银行,这是由于,一方面地方政府和地方股份制商业银行之间存在特殊的“政银关系”,受地方政府债务去杠杆政策的影响,部分股份制商业银行和中小型商业银行面临严重的经营压力。另一方面,部分股份制商业银行业务种类繁多,财务杠杆较高,同业业务的过度表外化形成了影子银行,具备风险生成银行和风险承受银行的双重特征,从而加速了金融风险跨行业、跨市场传递。因此监管部门需扩大银行资本金的补充工具,缓解银行流动性压力,并进一步限制股份制商业银行影子业务规模的快速扩张。

%后疫情时代,金融波动将持续存在,全球经济复苏的不确定性为国际金融系统稳定带来新的挑战,结合深度学习、人工智能等新工具、新方法来强化对系统性金融风险的监测,将有助于防范化解重大金融风险,避免“明斯基时刻”的到来。

\begin{thebibliography}{99}
    \bibitem{01}Diebold F X, Ylmaz K. On the network topology of variance decompositions: Measuring the connectedness of financial firms[J]. Journal of econometrics, 2014, 182(1): 119-134.
    \bibitem{02}陶玲,朱迎.系统性金融风险的监测和度量——基于中国金融体系的研究[J].金融研究,2016(06):18-36.
    \bibitem{03}Minsky H P. The Financial Instability Hypothesis: A Restatement. Post-Keynesian Economic Theory: A Challenge To Neo Classical Economics/Ed. by P. Arestis, T. Skouras, pp. 24-55. Brighton: ME Sharpe[J]. 1985.
    \bibitem{04}Kregel J A. Margins of safety and weight of the argument in generating financial fragility[J]. Journal of Economic Issues, 1997, 31(2): 543-548.
    \bibitem{05}IMF, BIS, FSB. 2010. “Guidance to Assess the Systemic Importance of Financial Institutions, Markets and Instruments
    Initial Considerations”- Background Paper.
    \bibitem{06}Corsetti, G. , P. Presenti and N. Roubini, 1998, "What Caused the Asian Currency and Financial Crisis?" Part I: A
    Macroeconomic Overview, Part II: The Policy Debate, mimeo, NYU.
    \bibitem{07}杨子晖,陈雨恬,林师涵.系统性金融风险文献综述:现状、发展与展望[J].金融研究,2022,499(1):185-217. 
    \bibitem{08}王美今,孙建军.中国股市收益、收益波动与投资者情绪[J].经济研究,2004(10):75-83.
    \bibitem{09}黄宏斌,翟淑萍,陈静楠.企业生命周期、融资方式与融资约束——基于投资者情绪调节效应的研究[J].金融研究,2016(07):96-112.
    \bibitem{10}张宗新,陈莹.系统性金融风险动态测度与跨部门网络溢出效应研究[J].国际金融研究,2022(01):72-84.
    \bibitem{11}Adrian T, Brunnermeier M K. CoVaR[J]. The American Economic Review, 2016, 106(7): 1705.
    \bibitem{12}杨子晖,陈雨恬,谢锐楷.我国金融机构系统性金融风险度量与跨部门风险溢出效应研究[J].金融研究,2018(10):19-37.
    \bibitem{13}Jorion P. Value at Risk: The New Benchmark for Controlling Market Risk, 2000.
    \bibitem{14}Acharya V, Pedersen L, Philippon T, et al. Measuring Systemic Risk, 2010.
    %\bibitem{15}宫晓琳.未定权益分析方法与中国宏观金融风险的测度分析[J].经济研究,2012,47(03):76-87.
    %\bibitem{16}唐文进,苏帆.极端金融事件对系统性风险的影响分析——以中国银行部门为例[J].经济研究,2017,52(04):17-33.
    \bibitem{15}李政,梁琪,涂晓枫.我国上市金融机构关联性研究——基于网络分析法[J].金融研究,2016(08):95-110.
    \bibitem{16}方意,郑子文.系统性风险在银行间的传染路径研究——基于持有共同资产网络模型[J].国际金融研究,2016(06):61-72.
    \bibitem{17}杨子晖,陈里璇,陈雨恬.经济政策不确定性与系统性金融风险的跨市场传染——基于非线性网络关联的研究[J].经济研究,2020,55(01):65-81.
    \bibitem{18}宫晓莉,熊熊,张维.我国金融机构系统性风险度量与外溢效应研究[J].管理世界,2020,36(08):65-83.
    \bibitem{19}马骏,何晓贝.金融风险传染机制研究——基于中国上市银行数据的模拟[J].金融研究,2021(09):12-29.
    \bibitem{20}党印,苗子清,张涛,冯冬发.大数据方法在系统性金融风险监测预警中的应用进展[J/OL].金融发展研究:1-10[2022-03-02].
    \bibitem{21}Gandrud C, Hallerberg M. What is a Financial Crisis? Efficiently Measuring Real-Time Perceptions of Financial Market Stress with an Application to Financial Crisis Budget Cycles[J]. Efficiently Measuring Real-Time Perceptions of Financial Market Stress with an Application to Financial Crisis Budget Cycles (November 30, 2015), 2015.
    \bibitem{22}Cerchiello P, Nicola G, Rönnqvist S, et al. Deep Learning for Assessing Banks' Distress from News and Numerical Financial Data[J]. Michael J. Brennan Irish Finance Working Paper Series Research Paper, 2018 (18-15).
    \bibitem{23}Chen M, Deininger M, Lee S J, et al. Identifying Financial Crises Using Machine Learning on Textual Data[J]. 2020.
    \bibitem{24}Nyman R, Kapadia S, Tuckett D. News and narratives in financial systems: exploiting big data for systemic risk assessment[J]. Journal of Economic Dynamics and Control, 2021, 127: 104119.
    \bibitem{25}Vaswani A, Shazeer N, Parmar N, et al. Attention is all you need[J]. Advances in neural information processing systems, 2017, 30.
    \bibitem{26}Alammar, J (2018). The Illustrated Transformer [Blog post]. Retrieved from https://jalammar.github.io/illustrated-transformer/
    \bibitem{27}Devlin J, Chang M W, Lee K, et al. Bert: Pre-training of deep bidirectional transformers for language understanding[J]. arXiv preprint arXiv:1810.04805, 2018.
    \bibitem{28}Cui Y, Che W, Liu T, et al. Pre-training with whole word masking for chinese bert[J]. IEEE/ACM Transactions on Audio, Speech, and Language Processing, 2021, 29: 3504-3514.
    \bibitem{29}The brainstem reticular formation is a small-world, not scale-free, network M. D. Humphries, K. Gurney and T. J. Prescott, Proc. Roy. Soc. B 2006 273, 503-511, doi:10.1098/rspb.2005.3354.
    \bibitem{30}Humphries and Gurney. Network 'Small-World-Ness': A Quantitative Method for Determining Canonical Network Equivalence. PLoS One. 2008, 3 (4). PMID 18446219. doi:10.1371/journal.pone.0002051.
    \bibitem{31}Telesford, Joyce, Hayasaka, Burdette, and Laurienti. The Ubiquity of Small-World Networks. Brain Connectivity. 2011, 1 (0038): 367-75. PMC 3604768. PMID 22432451. doi:10.1089/brain.2011.0038.
\end{thebibliography}
\end{document}